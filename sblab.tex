\documentclass[]{book}
\usepackage{lmodern}
\usepackage{amssymb,amsmath}
\usepackage{ifxetex,ifluatex}
\usepackage{fixltx2e} % provides \textsubscript
\ifnum 0\ifxetex 1\fi\ifluatex 1\fi=0 % if pdftex
  \usepackage[T1]{fontenc}
  \usepackage[utf8]{inputenc}
\else % if luatex or xelatex
  \ifxetex
    \usepackage{mathspec}
  \else
    \usepackage{fontspec}
  \fi
  \defaultfontfeatures{Ligatures=TeX,Scale=MatchLowercase}
    \setmainfont[]{Nanum Myeongjo}
\fi
% use upquote if available, for straight quotes in verbatim environments
\IfFileExists{upquote.sty}{\usepackage{upquote}}{}
% use microtype if available
\IfFileExists{microtype.sty}{%
\usepackage{microtype}
\UseMicrotypeSet[protrusion]{basicmath} % disable protrusion for tt fonts
}{}
\usepackage[margin=1in]{geometry}
\usepackage{hyperref}
\hypersetup{unicode=true,
            pdftitle={한국생물공학회 교육워크샵 {[}생물공학 x 기계학습{]}},
            pdfauthor={한국생명공학연구원 김하성},
            pdfborder={0 0 0},
            breaklinks=true}
\urlstyle{same}  % don't use monospace font for urls
\usepackage{natbib}
\bibliographystyle{apalike}
\usepackage{color}
\usepackage{fancyvrb}
\newcommand{\VerbBar}{|}
\newcommand{\VERB}{\Verb[commandchars=\\\{\}]}
\DefineVerbatimEnvironment{Highlighting}{Verbatim}{commandchars=\\\{\}}
% Add ',fontsize=\small' for more characters per line
\usepackage{framed}
\definecolor{shadecolor}{RGB}{248,248,248}
\newenvironment{Shaded}{\begin{snugshade}}{\end{snugshade}}
\newcommand{\KeywordTok}[1]{\textcolor[rgb]{0.13,0.29,0.53}{\textbf{#1}}}
\newcommand{\DataTypeTok}[1]{\textcolor[rgb]{0.13,0.29,0.53}{#1}}
\newcommand{\DecValTok}[1]{\textcolor[rgb]{0.00,0.00,0.81}{#1}}
\newcommand{\BaseNTok}[1]{\textcolor[rgb]{0.00,0.00,0.81}{#1}}
\newcommand{\FloatTok}[1]{\textcolor[rgb]{0.00,0.00,0.81}{#1}}
\newcommand{\ConstantTok}[1]{\textcolor[rgb]{0.00,0.00,0.00}{#1}}
\newcommand{\CharTok}[1]{\textcolor[rgb]{0.31,0.60,0.02}{#1}}
\newcommand{\SpecialCharTok}[1]{\textcolor[rgb]{0.00,0.00,0.00}{#1}}
\newcommand{\StringTok}[1]{\textcolor[rgb]{0.31,0.60,0.02}{#1}}
\newcommand{\VerbatimStringTok}[1]{\textcolor[rgb]{0.31,0.60,0.02}{#1}}
\newcommand{\SpecialStringTok}[1]{\textcolor[rgb]{0.31,0.60,0.02}{#1}}
\newcommand{\ImportTok}[1]{#1}
\newcommand{\CommentTok}[1]{\textcolor[rgb]{0.56,0.35,0.01}{\textit{#1}}}
\newcommand{\DocumentationTok}[1]{\textcolor[rgb]{0.56,0.35,0.01}{\textbf{\textit{#1}}}}
\newcommand{\AnnotationTok}[1]{\textcolor[rgb]{0.56,0.35,0.01}{\textbf{\textit{#1}}}}
\newcommand{\CommentVarTok}[1]{\textcolor[rgb]{0.56,0.35,0.01}{\textbf{\textit{#1}}}}
\newcommand{\OtherTok}[1]{\textcolor[rgb]{0.56,0.35,0.01}{#1}}
\newcommand{\FunctionTok}[1]{\textcolor[rgb]{0.00,0.00,0.00}{#1}}
\newcommand{\VariableTok}[1]{\textcolor[rgb]{0.00,0.00,0.00}{#1}}
\newcommand{\ControlFlowTok}[1]{\textcolor[rgb]{0.13,0.29,0.53}{\textbf{#1}}}
\newcommand{\OperatorTok}[1]{\textcolor[rgb]{0.81,0.36,0.00}{\textbf{#1}}}
\newcommand{\BuiltInTok}[1]{#1}
\newcommand{\ExtensionTok}[1]{#1}
\newcommand{\PreprocessorTok}[1]{\textcolor[rgb]{0.56,0.35,0.01}{\textit{#1}}}
\newcommand{\AttributeTok}[1]{\textcolor[rgb]{0.77,0.63,0.00}{#1}}
\newcommand{\RegionMarkerTok}[1]{#1}
\newcommand{\InformationTok}[1]{\textcolor[rgb]{0.56,0.35,0.01}{\textbf{\textit{#1}}}}
\newcommand{\WarningTok}[1]{\textcolor[rgb]{0.56,0.35,0.01}{\textbf{\textit{#1}}}}
\newcommand{\AlertTok}[1]{\textcolor[rgb]{0.94,0.16,0.16}{#1}}
\newcommand{\ErrorTok}[1]{\textcolor[rgb]{0.64,0.00,0.00}{\textbf{#1}}}
\newcommand{\NormalTok}[1]{#1}
\usepackage{longtable,booktabs}
\usepackage{graphicx,grffile}
\makeatletter
\def\maxwidth{\ifdim\Gin@nat@width>\linewidth\linewidth\else\Gin@nat@width\fi}
\def\maxheight{\ifdim\Gin@nat@height>\textheight\textheight\else\Gin@nat@height\fi}
\makeatother
% Scale images if necessary, so that they will not overflow the page
% margins by default, and it is still possible to overwrite the defaults
% using explicit options in \includegraphics[width, height, ...]{}
\setkeys{Gin}{width=\maxwidth,height=\maxheight,keepaspectratio}
\IfFileExists{parskip.sty}{%
\usepackage{parskip}
}{% else
\setlength{\parindent}{0pt}
\setlength{\parskip}{6pt plus 2pt minus 1pt}
}
\setlength{\emergencystretch}{3em}  % prevent overfull lines
\providecommand{\tightlist}{%
  \setlength{\itemsep}{0pt}\setlength{\parskip}{0pt}}
\setcounter{secnumdepth}{5}
% Redefines (sub)paragraphs to behave more like sections
\ifx\paragraph\undefined\else
\let\oldparagraph\paragraph
\renewcommand{\paragraph}[1]{\oldparagraph{#1}\mbox{}}
\fi
\ifx\subparagraph\undefined\else
\let\oldsubparagraph\subparagraph
\renewcommand{\subparagraph}[1]{\oldsubparagraph{#1}\mbox{}}
\fi

%%% Use protect on footnotes to avoid problems with footnotes in titles
\let\rmarkdownfootnote\footnote%
\def\footnote{\protect\rmarkdownfootnote}

%%% Change title format to be more compact
\usepackage{titling}

% Create subtitle command for use in maketitle
\providecommand{\subtitle}[1]{
  \posttitle{
    \begin{center}\large#1\end{center}
    }
}

\setlength{\droptitle}{-2em}

  \title{한국생물공학회 교육워크샵 {[}생물공학 x 기계학습{]}}
    \pretitle{\vspace{\droptitle}\centering\huge}
  \posttitle{\par}
    \author{한국생명공학연구원 김하성}
    \preauthor{\centering\large\emph}
  \postauthor{\par}
      \predate{\centering\large\emph}
  \postdate{\par}
    \date{2019-11-16}

\usepackage{booktabs}
\usepackage{amsthm}
\makeatletter
\def\thm@space@setup{%
  \thm@preskip=8pt plus 2pt minus 4pt
  \thm@postskip=\thm@preskip
}
\makeatother

\begin{document}
\maketitle

{
\setcounter{tocdepth}{1}
\tableofcontents
}
\hypertarget{docker}{%
\chapter{Docker}\label{docker}}

12월 생물공학 워크샵의 실습에 필요한 도커(Docker) 설치 및 사용법 입니다. 도커란 리눅스 컨테이너를 기반으로 한 오픈소스 가상화 플랫폼 입니다. 컨테이너는 격리된 리소스만을 이용하여 프로세스를 동작시키는 기술이며 Guest OS를 추가적으로 사용하는 기존 가상화 기술과 달리 OS 위에서 도커 엔진이 작동하고 그 위에서 프로세스만을 가상화하여 리소스를 적게 차지하고 빠르다는 장점이 있습니다. 또한 개발환경이 배포환경과 같아지므로 특정한 서비스를 개발하여 패키징하고 배포하는데 유용한 오픈소스 프로그램입니다.

컴퓨터에 여러 프로그램들을 설치하다 보면 가끔 하나의 프로그램 때문에 다른 프로그램의 실행에 문제가 생기는 경우가 종종 있습니다. 특히 리눅스 계열 OS를 포함한 오픈소스 패키지 기반의 R, python 등의 언어를 사용하다 보면 라이브러리들이 서로 얽히거나 같은 라이브러리를 사용하더라도 버전별로 의존성이 달라져서 컴퓨팅 환경이 지저분해지기 쉽습니다. 도커는 이러한 불편함을 줄여줄수 있습니다. 또한 다른 사람이 만들어 놓은 환경을 이미지 형태로 다운로드 받아서 그대로 사용할 수 있다는 점은 도커가 갖는 또다른 장점 중 하나 입니다. 실제로 머신러닝이나 딥러닝 연구를 수행할 때 기본 환경을 준비하는 과정이 가장 까다로운 작업일 수 있습니다.

본 워크샵에서는 python을 기반으로 딥러닝에 대한 실습을 진행하며 기본 환경은 docker와 anaconda를 이용하여 tensorflow, keras, scikit-learn, 그리고 jupyter-lab 환경으로 구성된 이미지를 다운받아 진행하고자 합니다. 따라서 아래 내용을 참고로 하여 지참하실 노트북에 docker 환경 구성을 완료한 후 워크샵에 참여 하시면 원할한 진행에 도움이 되겠습니다.

\hypertarget{docker-installation}{%
\section{Docker Installation}\label{docker-installation}}

도커를 윈도우즈에 설치하려면 \url{https://docs.docker.com/docker-for-windows/} 를 참고해서 다음 단계별로 수행하시면 됩니다.

\begin{itemize}
\tightlist
\item
  Docker Hub (\url{https://hub.docker.com/}) Get Started 클릭 후 사용자 등록 (계정이 없을 경우)
\item
  \url{https://www.docker.com/products/docker-desktop} 에서 Download Desktop for Mac and Windows 클릭
\item
  튜토리얼 창이 열리면 Download Docker Desktop for Windows 클릭 (튜토리얼은 설치 후 따라서 실습해 보셔도 좋습니다)
\item
  설치 파일 다운로드 확인후 파일 실행 (\url{https://docs.docker.com/docker-for-windows/install/} 참고)
\item
  안내에 따라서 설치 진행, 마지막 `Finish' 클릭
\end{itemize}

\hypertarget{docker-images}{%
\section{Docker Images}\label{docker-images}}

도커를 사용하기 위해서는 다른 사람이 만들어 둔 이미지를 다운로드 받아서 사용하거나 직접 Dockerfile을 편집하여 필요한 소프트웨어를 설치하고 이미지를 만들어 사용할 수 있습니다.

\begin{Shaded}
\begin{Highlighting}[]
\ExtensionTok{docker}\NormalTok{ build -t haseong/bioeng-ml:0 .}
\end{Highlighting}
\end{Shaded}

\begin{Shaded}
\begin{Highlighting}[]
\NormalTok{x }\OperatorTok{=} \DecValTok{10}
\ControlFlowTok{for}\NormalTok{ i }\KeywordTok{in} \BuiltInTok{range}\NormalTok{(}\DecValTok{10}\NormalTok{):}
\NormalTok{  x }\OperatorTok{=}\NormalTok{ i}
\BuiltInTok{print}\NormalTok{(x)}
\end{Highlighting}
\end{Shaded}

\includegraphics{images/01-16.PNG}

\bibliography{book.bib,packages.bib}


\end{document}
